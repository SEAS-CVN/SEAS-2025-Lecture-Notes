\documentclass[11pt]{article}
\usepackage{amsfonts,amsthm,amsmath,amssymb, hyperref, dsfont, enumitem, bbm}
\usepackage{array}
\usepackage{epsfig}
\usepackage{fullpage}
\usepackage{color}
\usepackage{epigraph}
\renewcommand{\epigraphflush}{center}
\usepackage[framemethod=tikz]{mdframed}
\usepackage{titlesec}
\usepackage{ellipsis}
\usepackage{subcaption}
\usepackage[normalem]{ulem}
\usepackage{todonotes}


\titleformat{\chapter}[display]
  {\normalfont\bfseries}{}{0pt}{\Huge}

\usepackage{clrscode3e,etoolbox}
%%%%%%%%%%%%%%%%%%%%%%%%%%%%%%%%%%%%%%%%%%%%%%%%%%%%%%%%%%%%%%%%%%%%%%
% Structure
%%%%%%%%%%%%%%%%%%%%%%%%%%%%%%%%%%%%%%%%%%%%%%%%%%%%%%%%%%%%%%%%%%%%%%

% Don't put chapter number in section numbers
% \renewcommand*\thesection{\arabic{section}}	% commented out to include chapter number in section numbers

\newcommand{\anchorednote}[2]{ #1 \note{#2} }	% anchored note - needed for latex2edx
\newcommand{\note}[1]{\todo[color=blue!10,
  linecolor=blue!90,size=\small]{\linespread{0.9}\selectfont{#1}\par}}

\usepackage{tcolorbox}
\newtcolorbox{examplebox}{colback=green!5!white}

% \scalebox{1.5}{\begin{tikzpicture}
%   % Put your tikz code here
% \end{tikzpicture}}

\newcommand\question[1]{\vskip0.05in\todo[inline, color=yellow!5]{{\bf
      Study Question:}  #1}} 

\def\myrightmargin{2.0in}

% Make it compact for printing
\renewcommand{\note}[1]{\footnote{#1}}
\def\myrightmargin{1.0in}
\usepackage[left=1in, top=1in, bottom=1in,right=\myrightmargin]{geometry}



%%%%%%%%%%%%%%%%%%%%%%%%%%%%%%%%%%%%%%%%%%%%%%%%%%%%%%%%%%%%%%%%%%%%%%
% Math macros
%%%%%%%%%%%%%%%%%%%%%%%%%%%%%%%%%%%%%%%%%%%%%%%%%%%%%%%%%%%%%%%%%%%%%%

% Use to index over examples

\newcommand\ex[2]{#1^{(#2)}}
% Data sets
\newcommand\data{{\cal D}}
\newcommand\dataTrain{{\cal D}_n}
\newcommand\dataTest{{\cal D}_{n'}}
% Model, hypoth
\newcommand\model{{\cal M}}
\newcommand\hclass{{\cal H}}
% Max likelihood
\newcommand\ml[1]{#1_{\bf ml}}
% Empirical risk min
\newcommand\erm[1]{#1_{\bf erm}}
% Arg max
\newcommand\argmax[1]{{\rm arg}\max_{#1}}
\newcommand\argmin[1]{{\rm arg}\min_{#1}}
% Math
\newcommand{\R}{\mathbb{R}}
% errors
\newcommand{\trainerr}{\mathcal{E}_n}
\newcommand{\testerr}{\mathcal{E}}
% sign
\newcommand{\sign}{\text{sign}}
% 2d vec
\newcommand*{\twodrow}[2]{\begin{bmatrix} #1 & #2 \end{bmatrix}}
\newcommand*{\twodcol}[2]{\begin{bmatrix} #1 \\ #2 \end{bmatrix}}
% norm


\newcommand{\qn}[1]{\textcolor{orange}{Quynh: #1}}


\begin{document}

%% Courtesy: Daniel Spielman, via Madhu Sudan --> Chi-Ning Chou --> Anurag Anshu

\theoremstyle{plain}
\newtheorem{theorem}{Theorem}[section]
\newtheorem{lemma}[theorem]{Lemma}
\newtheorem{example}[theorem]{Example}
\newtheorem{corollary}[theorem]{Corollary}
\theoremstyle{definition}
\newtheorem{definition}[theorem]{Definition}
\newtheorem*{mydefinition}{Definition}
\newtheorem{claim}[theorem]{Claim}
\newtheorem{fact}[theorem]{Fact}
\newtheorem{remark}[theorem]{Remark}
\newtheorem{exercise}[theorem]{Exercise}

%Left and right brackets
\newcommand {\br} [1] {\ensuremath{ \left( #1 \right) }}
\newcommand {\Br} [1] {\ensuremath{ \left[ #1 \right] }}


%Quantum notations
\newcommand {\norm}[1]{{\| #1 \|}}  
\newcommand {\bra} [1] {\ensuremath{ \left\langle #1 \right| }}
\newcommand {\ket} [1] {\ensuremath{ \left| #1 \right\rangle }}
\newcommand {\ketbratwo} [2] {\ensuremath{ \left| #1 \middle\rangle \middle\langle #2 \right| }}
\newcommand {\ketbra} [1] {\ketbratwo{#1}{#1}}
\newcommand{\braket}[2]{\langle#1|#2\rangle}
\newcommand{\Tr}[1]{\mathrm{Tr}\left(#1\right)}
\newcommand{\tr}[2]{\mathrm{Tr}_{#1}\left(#2\right)}
\newcommand{\PE}{\mathrm{PE}}
\newcommand{\EPR}{\mathrm{EPR}}
\newcommand{\CNOT}{\mathrm{CNOT}}
\newcommand{\CZ}{\mathrm{CZ}}

%Generic math symbols
\newcommand {\eps} {\varepsilon}
\newcommand{\tO}{\tilde{O}}
\newcommand{\ind}[1]{\mathrm{Ind}\left(#1\right)}
\newcommand {\id}{\mathds{1}}
\newcommand{\bitn}[1]{\ensuremath{\{0,1\}^{#1}}}
\newcommand{\bitone}{\ensuremath{\{0,1\}}}
\newcommand{\swp}{\mathrm{\textbf{Swap}}}

% Information theory and CS symbols
\newcommand {\prob} {\ensuremath{\mathrm{Prob}}}
\newcommand{\bigo}[1]{\mathcal{O}\left(#1\right)}
\newcommand{\omeg}[1]{\Omega\left(#1\right)}
\newcommand{\expec}{\mathbb{E}}
\newcommand{\relent}[2]{\mathrm{D}\left(#1\|#2\right)}
\newcommand{\KL}[2]{\mathrm{D}_{KL}\left(#1\|#2\right)}
\newcommand{\mutinf}[2]{\mathrm{I}\left(#1:#2\right)}
\newcommand{\condmutinf}[3]{\mathrm{I}\left(#1:#2|#3\right)}
\newcommand{\condent}[2]{\mathrm{S}\left(#1|#2\right)}
\newcommand{\ent}[1]{\mathrm{S}\left(#1\right)}
\newcommand{\cla}{\text{classical}}
\newcommand{\qua}{\text{quantum}}
\newcommand{\sym}{\textsf{sym}}

%Statistical measures
\newcommand{\F}{\mathrm{F}}
\newcommand{\tv}{\mathrm{TV}}
\newcommand{\Hol}{\mathrm{Hol}}
\newcommand{\hel}{\mathrm{Hd}}
\newcommand{\pur}{\mathrm{Pur}}
\newcommand{\scs}{\mathrm{succ}}
\newcommand{\err}{\mathrm{err}}
\newcommand{\can}{\mathrm{can}}
\newcommand{\Var}{\mathrm{Var}}

%Fancy alphabets
\def\cA{\mathcal{A}}
\def\cB{\mathcal{B}}
\def\cC{\mathcal{C}}
\def\cD{\mathcal{D}}
\def\cE{\mathcal{E}}
\def\cF{\mathcal{F}}
\def\cG{\mathcal{G}}
\def\cH{\mathcal{H}}
\def\cL{\mathcal{L}}
\def\cM{\mathcal{M}}
\def\cO{\mathcal{O}}
\def\cP{\mathcal{P}}
\def\cR{\mathcal{R}}
\def\cS{\mathcal{S}}
\def\cT{\mathcal{T}}
\def\cX{\mathcal{X}}
\def\N{\mathbb{N}}
\def\Z{\mathbb{Z}}



%%%%%%%%%%%%%%%%%%%%%%%%%%%%%%%%%%%%%%%%%%%%% 
%Commands below can be ignored by the students  
%%%%%%%%%%%%%%%%%%%%%%%%%%%%%%%%%%%%%%%%%%%%%
% Header
\newcommand{\handout}[5]{
   \renewcommand{\thepage}{#1-\arabic{page}}
   \noindent
   \begin{center}
   \framebox{
      \vbox{
    \hbox to 6.3in { {\bf #1}
     	 \hfill {\it #3} }
       \vspace{4mm}
       \hbox to 6.3in { {\Large \hfill #5  \hfill} }
       \vspace{2mm}
       \hbox to 6.3in { {\it #2 \hfill #4} }
      }
   }
   \end{center}
   \vspace*{4mm}
}

\newcommand{\lecture}[4]{\handout{#1}{#2}{Lecturer:
#3}{Scribe: #4}{Lecture #1}}



%Editing commands
\newcommand{\edit}[2]{\st{#1}\hspace{0.05in}\textcolor{blue}{#2}}
\newcommand{\suppress}[1]{}

\newcommand{\anote}[1]{{\color{red} \textbf{Anurag's note:} #1}}
\newcommand{\qnote}[1]{{\color{red} \textbf{Quynh's note:} #1}}
\newcommand{\hkn}[1]{{\color{orange} \textbf{Nguyen nu:} #1}}


%Itemizing and Equation shorthands
\newcommand{\beit}{\begin{itemize}}
\newcommand{\enit}{\end{itemize}}
\newcommand{\been}{\begin{enumerate}}
\newcommand{\enen}{\end{enumerate}}
\newcommand{\beq}{\begin{equation}}
\newcommand{\enq}{\end{equation}}
\newcommand{\beqst}{\begin{equation*}}
\newcommand{\enqst}{\end{equation*}}
\newcommand{\beqar}{\begin{eqnarray}}
\newcommand{\enqar}{\end{eqnarray}}
\newcommand{\beqarst}{\begin{eqnarray*}}
\newcommand{\enqarst}{\end{eqnarray*}}




\handout{SEAS 2025}{{\bf July 22, 2025}}{Instructor: Quynh Nguyen}{TA: }{Lecture 1: Overview of Machine Learning}

\emph{Note}: Based on MIT 6.036 - Intro to Machine Learning (Fall 2019) -- Adapted to SEAS by Nguyen Hoang, Nguyen Huynh, Hoang Nguyen, Quynh Nguyen, Nam Tran

\section{AI, ML and DL}
\begin{itemize}
    \item Definition
    \item Example
\end{itemize}
\hkn{Provide a brief overview to get a sense of their potential and the relationship of these three fields}
\section{Machine Learning}
The main focus of machine learning is {\em making decisions or
  predictions based on data}.  There are a number of other fields with
significant overlap in technique, but difference in focus: \note{This
  description paraphrased from a post on 9/4/12 at {\tt andrewgelman.com}}
in economics and psychology, the goal is to discover underlying causal
processes and in statistics it is to find a model that fits a data set
well.  In those fields, the end product is a model.  In machine
learning, we often fit models, but as a means to the end of making
good predictions or decisions.

As machine-learning (ML) methods have improved in their capability and
scope, ML has become the best way, measured in terms of speed, human
engineering time, and robustness, to approach many applications.  Great
examples are face detection and speech recognition and many kinds of
language-processing tasks.   Almost any application that involves
understanding data or signals that come from the real world can be
best addressed using machine learning.

One crucial (and often undervalued) aspect of machine
learning approaches to solving problems is that human engineering
plays an important role.  A human still has to {\em frame} the
problem:  acquire and organize data, design a space of possible
solutions, select a learning algorithm and its parameters, apply the
algorithm to the data, validate the resulting solution to decide
whether it's good enough to use, etc.   These steps are of great
importance.  

% Generally, this is done in two stages:
% \begin{enumerate}
% \item {\bf Learn} or {\bf estimate} a model from the data
% \item {\bf Apply} the model to make predictions or answer queries
% \end{enumerate}

The conceptual basis of learning from data is the {\em problem of
  induction}: %\note{Bertrand Russell is my hero. --lpk} 
  Why do we think that
previously seen data will help us predict the future?  This is a
serious philosophical problem of long standing.  We will
operationalize it by making assumptions, such as that all training
data are IID (independent and identically distributed)\note{IID stands for {\em independent and identically distributed},
  which means that the elements in the set are related in the sense
  that they all come from the same underlying probability
  distribution, but not in any other ways.} and that
queries will be drawn from the same distribution as the training data,
or that the answer comes from a set of possible answers known in
advance.

In general, we need to solve these two problems:
\begin{itemize}
\item {\bf Estimation:}  When we have data that are noisy reflections
  of some underlying quantity of interest, we have to aggregate the
  data and make estimates or predictions about the quantity.
  How do we deal with the fact that, for
  example, the same treatment may end up with different results on
  different trials?  How can we predict how well an estimate may
  compare to future results?
\item {\bf Generalization:} How can we predict results of a situation
  or experiment that we have never encountered before in our data set?
\end{itemize}

We can describe problems and their solutions using six
characteristics, three of which characterize the problem and three of
which characterize the solution:
\begin{enumerate}
\item {\bf Problem class:}  What is the nature of the training data and what
  kinds of queries will be made at testing time?
\item {\bf Assumptions:}  What do we know about the source of the data or the form of the solution?
\item {\bf Evaluation criteria:}  What is the goal of the prediction or estimation system?  How  will the answers to individual queries be evaluated?  How will the overall performance of the system be measured?
\item {\bf Model type:}  Will an intermediate model be made?  What aspects of the data will be modeled?  How will the model be used to make predictions?
\item {\bf Model class:} What particular parametric class of models
  will be used?  What criterion will we use to pick a particular model
  from the model class?
\item {\bf Algorithm:}  What computational process will be used to fit
  the model to the data and/or to make predictions?
\end{enumerate}
Without making some assumptions about the nature of the process
generating the data, we cannot perform generalization.  In the
following sections, we elaborate on these ideas.  



\section{Supervised learning vs unsupervised learning}

\hkn{This section 3 quite overlaps with beginning of ML 2, I think instead it should be reworded to highlight more about \textbf{task and dataset}, since subsequent sections would be about the evaluation metric, model, algorithm. This will make the flow more consistent as \textbf{components of a ML problem}}

There are many different {\em problem classes} in machine learning.
They vary according to what kind of data is provided and what kind of
conclusions are to be drawn from it.  

Five standard problem classes are described below, to establish some notation and
terminology. In this course, we will focus on classification and regression (two
examples of supervised learning), and we will touch on reinforcement
learning, sequence learning, and clustering. 

\begin{remark}
Don't feel you
  have to memorize all these kinds of learning, etc.  We just want you
  to have a very high-level view of (part of) the breadth of the
  field.    
\end{remark}
  


\subsection{Supervised learning}

The idea of {\em supervised} learning is that the learning system is
given inputs and told which specific outputs should be associated with
them.  We divide up supervised learning based on whether the outputs
are drawn from a small finite set (classification) or a large finite
or continuous set (regression).

\subsubsection{Regression}

For a regression problem, the training data $\dataTrain$ is in the form of a set of pairs $\{(\ex{x}{1},
\ex{y}{1}), \ldots, (\ex{x}{n}, \ex{y}{n})\}$ where $\ex{x}{i}$
represents an input, most typically a
$d$-dimensional vector of real and/or discrete values, and
$\ex{y}{i}$ is the output to be predicted, in this case a real-number.
\note{Many
  textbooks use $x_i$ and $t_i$ instead of $\ex{x}{i}$ and
  $\ex{y}{i}$.  We find that notation somewhat difficult to manage
  when $\ex{x}{i}$ is itself a vector and we need to talk about its
  elements.  The notation we are using is standard in some other parts
  of the machine-learning literature.}  The $y$ values are sometimes
called {\em target values}.

The goal in a regression problem is ultimately, given a new input 
value $\ex{x}{n+1}$, to predict the value of $\ex{y}{n+1}$.
Regression problems are a kind of {\em supervised learning},
because the desired output $\ex{y}{i}$ is specified for
each of the training examples $\ex{x}{i}$.

\subsubsection{Classification}

A classification problem is like regression, except that 
$\ex{y}{i}$ is discrete.  The classification problem is 
{\em binary} or {\em two-class} if $\ex{y}{i}$ (also known as the {\em class})
is drawn from a set of two possible values; If there are more than two classes, it is called
{\em multi-class}. Note that in regression, the model can predict continuous values that were not seen in the training set, but in classification, the model can only predict class labels that were present during training.

\subsection{Unsupervised learning}
\label{sec:intro_unsupervised}

{\em Unsupervised} learning doesn't
involve learning a function from inputs to outputs based on a set of
input-output pairs.  Instead, one is given a data set and generally
expected to find some patterns or structure inherent in it.

\subsubsection{Density estimation}

% switched from y -> x, to be consistent with other examples
Given samples $\ex{x}{1}, \ldots, \ex{x}{n} \in \R^d$ drawn IID 
from some distribution $\Pr(X)$, the goal is to predict the probability
$\Pr(\ex{x}{n+1})$ of an element drawn from the same distribution.
Density estimation sometimes plays a role as a ``subroutine'' in the
overall learning method for supervised learning, as well.


\subsubsection{Clustering}

Given samples $\ex{x}{1}, \ldots, \ex{x}{n} \in \R^d$, the goal is to
find a partitioning (or ``clustering'') of the samples that groups together samples that are
similar.  There are many different objectives, depending on the
definition of the similarity between samples and exactly what
criterion is to be used (e.g., minimize the average distance between
elements inside a cluster and maximize the average distance between
elements across clusters).  Other methods perform a ``soft''
clustering, in which samples may be assigned 0.9 membership in one
cluster and 0.1 in another.  Clustering is sometimes used as a step in
density estimation, and sometimes to find useful structure in data.

\subsubsection{Dimensionality reduction}

Given samples $\ex{x}{1}, \ldots, \ex{x}{n} \in \R^D$, the problem is to
re-represent them as points in a $d$-dimensional space, where $d <
D$.  The goal is typically to retain information in the data set that
will, e.g., allow elements of one class to be discriminated from
another.

Dimensionality reduction is a standard technique which is particularly useful for
visualizing or understanding high-dimensional data.  If the goal is
ultimately to perform regression or classification on the data after
the dimensionality is reduced, it is usually best to articulate an objective
for the overall prediction problem rather than to first do
dimensionality reduction without knowing which dimensions will be
important for the prediction task.

\hkn{I suggest dropping 3.3-3.5 if it's not mentioned or elaborated again in subsequent lectures. I fear may overwhelm information while only touching on a superficial level. }
\subsection{Reinforcement learning}

In reinforcement learning, the goal is to learn a mapping from input
values $x$ to output values $y$, but without a direct supervision
signal to specify which output values $y$ are best for a particular
input.  There is no training set specified {\em a priori}.  Instead,
the learning problem is framed as an agent interacting with an
environment, in the following setting:
\begin{itemize}
\item The agent observes the current state, $\ex{x}{0}$.
\item It selects an action, $\ex{y}{0}$.
\item It receives a reward, $\ex{r}{0}$, which depends on $\ex{x}{0}$
  and possibly $\ex{y}{0}$.
\item The environment transitions probabilistically to a new state,
  $\ex{x}{1}$, with a distribution that depends only on $\ex{x}{0}$
  and $\ex{y}{0}$. 
\item The agent observes the current state, $\ex{x}{1}$.
\item $\ldots$
\end{itemize}
The goal is to find a policy $\pi$, mapping $x$ to $y$, (that is,
states to actions) such that some long-term sum or average of rewards
$r$ is maximized.

This setting is very different from either supervised learning or
unsupervised learning, because the agent's action choices affect both
its reward and its ability to observe the environment.  
It requires careful consideration of the long-term effects of actions,
as well as all of the other issues that pertain to supervised
learning.

\subsection{Sequence learning}

% allow output sequence to be different length from input seq
In sequence learning, the goal is to learn a mapping from {\em input
  sequences} $x_0, \ldots, x_n$ to {\em output sequences} $y_1,
\ldots, y_m$.  The mapping is typically represented as a {\em state
  machine}, with one function $f$ used to compute the next hidden
internal state given the input, and another function $g$ used to
compute the output given the current hidden state.  

It is supervised in the sense that we are told what output sequence to
generate for which input sequence, but the internal functions have to
be learned by some method other than direct supervision, because we
don't know what the hidden state sequence is.


\qn{below subsection may be omitted?}
\subsection{Other settings}
There are many other problem settings.  Here are a few.

In {\em semi-supervised} learning, we have a supervised-learning training set, but there may be an additional set of
$\ex{x}{i}$ values with no known $\ex{y}{i}$.  These values can still
be used to improve learning performance if they are drawn from
$\Pr(X)$ that is the marginal of $\Pr(X, Y)$ that governs the rest of
the data set.

In {\em active} learning, it is assumed to be expensive to acquire a
label $\ex{y}{i}$ (imagine asking a human to read an x-ray image), so 
the learning algorithm can sequentially ask for particular inputs
$\ex{x}{i}$ to be labeled, and must carefully select queries in order
to learn as effectively as possible while minimizing the cost of
labeling. 

In {\em transfer} learning (also called {\em meta-learning}), there
are multiple tasks, with data drawn 
from different, but related, distributions.  The goal is for
experience with previous tasks to apply to learning a current task in
a way that requires decreased experience with the new task.


\section{Assumptions}

The kinds of assumptions that we can make about the data source or the solution include:
\begin{itemize}
\item The data are independent and identically distributed.
\item The data are generated by a Markov chain.
\item The process generating the data might be adversarial.
\item The ``true'' model that is generating the data can be perfectly described by one of some particular set of hypotheses.
\end{itemize}
The effect of an assumption is often to reduce the ``size'' or
``expressiveness''  of the space
of possible hypotheses and therefore reduce the amount of data
required to reliably identify an appropriate hypothesis.

% an explicit example of how such assumptions affect one of the above learning classes could be helpful here

\section{Evaluation criteria}

Once we have specified a problem class, we need to say what makes an
output or the answer to a query good, given the training data.  We specify evaluation criteria at two levels:  how an individual prediction is scored, and how the overall behavior of the prediction or estimation system is scored.

The quality of predictions from a learned model is often
expressed in terms of a {\em loss function}.  A loss function $L(g,
a)$ tells you how much you will be penalized for making a guess
$g$ when the answer is actually $a$.  There are many possible loss
functions.  Here are some frequently used examples:
\begin{itemize}
\item {\bf 0-1 Loss} applies to predictions drawn from finite
  domains.\note{If the actual values are drawn from a continuous
    distribution, the probability they would ever be equal to some
    predicted $g$ is 0 (except for some weird cases).}
\[L(g, a) = \begin{cases}
         0 & \text{if $g = a$} \\
         1 & \text{otherwise}
\end{cases}\]
\item {\bf Squared loss}
\[L(g, a) = (g - a)^2\]
\item {\bf Linear loss}
\[L(g, a) = |g - a|\]
\item {\bf Asymmetric loss}
Consider a situation in which  you are trying to predict whether
someone is having a heart attack.  It might be much worse to predict
``no'' when the answer is really ``yes'', than the other way around.
\[L(g, a) = \begin{cases}
         1 & \text{if $g = 1$ and $a = 0$} \\
         10 & \text{if $g = 0$ and $a = 1$} \\
         0 & \text{otherwise}
\end{cases}\]
\end{itemize}

Any given prediction rule will usually be evaluated based on multiple predictions and the loss of each one.  At this level, we might be interested in:
\begin{itemize}
\item Minimizing expected loss over all the predictions (also known as risk)
\item Minimizing maximum loss: the loss of the worst prediction
\item Minimizing or bounding regret: how much worse this predictor performs than the best one drawn from some class
\item Characterizing asymptotic behavior: how well the predictor will perform in the limit of infinite training data
\item Finding algorithms that are probably approximately correct:  they probably generate a hypothesis that is right most of the time.
\end{itemize}

There is a theory of rational agency that argues that you should
always select the action that {\em minimizes the expected loss}.  This
strategy will, for example, make you the most money in the long run,
in a gambling setting.  \note{Of course, there are other models for
  action selection and it's clear that people do not always (or maybe
  even often) select actions that follow this rule.}  Expected loss is
also sometimes called {\em risk} in the machine-learning literature,
but that term means other things in economics or other parts of
decision theory, so be careful...it's risky to use it.  We will, most of the time, concentrate on this criterion.


\section{Model type}

Recall that the goal of a ML system is typically to
estimate or generalize, based on data provided.  Below, we examine the
role of model-making in machine learning.

\subsection{No model}

In some simple cases, in response to queries, we can generate
predictions directly from the training data, without the construction
of any intermediate model.  For example, in regression or
classification, we might generate an answer to a new query by
averaging answers to recent queries, as in the {\em nearest neighbor}
method.

\subsection{Prediction rule}

This two-step process is more typical:
\begin{enumerate}
\item ``Fit'' a model to the training data
\item Use the model directly to make predictions
\end{enumerate}

In the {\em prediction rule} setting of regression or classification, the
model will be some hypothesis or prediction rule $y = h(x ; \theta)$
for some functional form $h$.  The idea is that $\theta$ is a vector
of one or more parameter values that will be determined by fitting the
model to the training data and then be held fixed.  Given a new
$\ex{x}{n+1}$, we would then make the prediction $h(\ex{x}{n+1};
\theta)$.  \note{We write $f(a ; b)$ to describe a function that is
  usually applied to a single argument $a$, but is a member of a
  parametric family of functions, with the particular function
  determined by parameter value $b$.  So, for example, we might write
  $h(x ; p) = x^p$ to describe a function of a single argument that is
  parameterized by $p$.}

The fitting process is often articulated as an optimization problem:
Find a value of $\theta$ that minimizes some criterion involving
$\theta$ and the data.
An optimal strategy, if we knew the actual underlying distribution on
our data, $\Pr(X,Y)$ would be to predict the value of $y$ that
minimizes the {\em expected loss}, which is also known as the {\em
  test error}.  If we don't have that actual underlying distribution, or
even an estimate of it, we can take the approach of minimizing the
{\em training error}: that is, finding the prediction rule $h$ that
minimizes the average loss on our training data set.  So, we would
seek $\theta$ that minimizes
\[\trainerr(\theta) =  \frac{1}{n}\sum_{i = 1}^n
L(h(\ex{x}{i};\theta), \ex{y}{i})\;\;,\]
where the loss function $L(g, a)$ measures how bad it would be to make
a guess of $g$ when the actual value is $a$.

We will find that minimizing training error alone is often not a good
choice: it is possible to emphasize fitting the current data too
strongly and end up with a hypothesis that does not generalize well
when presented with new $x$ values.

% why are only two model types discussed?

\section{Model class and parameter fitting}
\label{modelClass}

A model {\em class} $\model$ is a set of possible models, typically  parameterized by a vector of parameters $\Theta$.
What assumptions will we make about the form of the model?  When
solving a regression problem using a prediction-rule approach, we
might try to find a linear function $h(x ; \theta, \theta_0) = \theta^T x + \theta_0$
that fits our data well.  In this example, the parameter vector
$\Theta = (\theta, \theta_0)$.

For problem types such as discrimination and classification, there
are huge numbers of model classes that have been considered...we'll
spend much of this course exploring these model classes, especially
neural networks models.
We will almost completely restrict our attention to model classes with
a fixed, finite number of parameters.  Models that 
relax this assumption are called ``non-parametric'' models.

How do we select a model class?  In some cases, the machine-learning
practitioner will have a good idea of what an appropriate model class
is, and will specify it directly.  In other cases, we may consider
several model classes.  In such situations, we are solving a {\em
  model selection} problem: model-selection is to pick a model class
$\model$ from a (usually finite) set of possible model classes; {\em
  model fitting} is to pick a particular model in that class,
specified by parameters $\theta$.

\section{Algorithm}
Once we have described a class of models and a way of scoring a model
given data, we have an algorithmic problem: what sequence of
computational instructions should we run in order to find a good model
from our class?  For example, determining the parameter vector
$\theta$ which minimizes $\trainerr(\theta)$ might be done using a
familiar least-squares minimization algorithm, when the model $h$ is a
function being fit to some data $x$.

Sometimes we can use software that was designed, generically, to
perform optimization.  In many other cases, we use algorithms that are
specialized for machine-learning problems, or for particular
hypotheses classes.
Some algorithms are not easily seen as trying to optimize a particular
criterion.  In fact, a historically important method for finding linear
classifiers, the perceptron algorithm, has this character.



\end{document}